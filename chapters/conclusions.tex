\chapter{Conclusions}

In this work we have established existence of expanders
in power-law graphs under several models,
and have studied their behavior for different ranges of the exponent $\beta$.
We have also made an overview of the structure of random graphs,
comparing them side by side with our results
to contrast some correspondences and differences.
In particular, we have shown that power-law graphs with small $\beta$
have similar to $G(n,p)$ expansion properties, just as one would expect.
Also the largest components of power-law graphs have size comparable
to the size of our edge expanders, so these components are well-connected.

Possible directions for future research include tightening gaps
between sizes of connected components and expanding subgraphs,
and also developing improved SAT algorithms for power-law formulas,
which would take advantage of this extra structural information.
Self-similarity might be exploited to design recursive algorithms
for certain families of SAT formulas.
Another promising practical result would be
to connect spectral and combinatorial expansion for power-law graphs.
It would be as well interesting to improve the quality of expansion
and weaken the conditions for having large expanding sets of size up to $n/2$
instead of just the small ones.
Finally, the results about random walks with lookahead suggest studying
the expansion of power-law graphs with new edges added for each path
of some small constant length between every pair of vertices.
