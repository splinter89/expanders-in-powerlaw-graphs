\chapter{Conclusions}

In this work we established existence of expanders in power-law graphs
under several models and studied their behavior for various ranges
of the exponent $\beta$.
We also made an overview of the structure of random graphs, comparing them
side by side with our findings to contrast some correspondences and differences.
In particular, we showed that power-law graphs with small $\beta$
have similar to $G(n,p)$ expansion properties, just as one would expect.
Also the~largest components of power-law graphs and our edge expanders
have comparable sizes, so these components are presumably well-connected.

One possible direction for future research is the improvement of the current results.
This includes tightening gaps between sizes of connected components and expanding subgraphs,
increasing the quality of expansion,
and weakening the conditions for having large expanding sets of size up to $n/2$.
It would be useful to connect spectral and combinatorial expansion of power-law graphs.
By analogy with connectivity in percolation theory, interesting open problem is
to decide the expansion of sparse graphs obtained by randomly removing each edge
with probability proportional to the degrees of its endpoints.

Another promising direction is  the development of enhanced SAT algorithms
for power-law formulas, facilitated by the available structural information.
Self-similarity of a power law might be exploited to design recursive SAT algorithms.
Finally, the results about random walks with lookahead suggest studying
the expansion of $k$-th power of power-law graphs, for some small constant~$k$.
