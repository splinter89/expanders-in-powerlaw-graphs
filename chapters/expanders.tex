\chapter{Expanders in Power-Law Graphs}
\label{ch:expanders}

From now on, we will work with an induced subgraph $H=(V_H,E_H)$ of $G$,
obtained by retaining only vertices of degree at least $d_0$.
This definition is both intuitive and easy to work with.
We denote the size $|V_H|=n'$.

If $H$ is an expander, then outputting $H$ is as simple as checking
the degree of each vertex during the process of generation of the graph $G$.
Connecting large spectral gap with high combinatorial expansion
would allow one to eliminate error of outputting non-expander,
but it is still an open problem for our models.

\section{Coin Toss Model}

\begin{theorem}[Existence of an edge expander]
    \label{thm:powerlaw-coin-toss-edge-expansion}
    $\exists d_0\;\forall c<1/3\;\forall\delta>0\;\exists\gamma,c_1=c_1(\beta,c)>0:$
    let $G=(V,E)$ be a random power-law graph
    and $H=(V_H,E_H)$ be its induced subgraph on $|V_H|=n'$ vertices,
    such that $V_H=\{v\in V\;|\;\deg(v)\geq d_0\}$.
    
    Then if $0<\beta<1$, the whole graph $G$ is $(n/2,\gamma)$ edge expander w.h.p.
    
    Otherwise, the subgraph $H$ is $(n'/2,\gamma)$ edge expander w.h.p.
    \begin{gather}
        d_{max}=\max_{v\in V}{\deg(v)}\\
        d_0=
        \begin{cases}
            1 & \quad \text{if } 0<\beta<1,\\
            d_{max}/\sqrt{n} & \quad \text{if } \beta=1,\\
            \frac{d_{max}}{n^{1/\beta}} & \quad \text{if } 1<\beta\leq 1.6,\\
            c\,d_{max} & \quad \text{if } \beta>1.6.
        \end{cases}
    \end{gather}
    \begin{gather}
        n'=
        \begin{cases}
            n & \quad \text{if } 0<\beta<1,\\
%            \left(\ln 1/c\right)\frac{n}{\ln n}=\Theta\left(\frac{n}{\ln n}\right) & \quad \text{if } \beta=1,\\
            n/2 & \quad \text{if } \beta=1,\\
            \frac{n}{(\beta-1)\,\zeta(\beta)^{1/\beta}} & \quad \text{if } 1<\beta\leq 1.6,\\
            \frac{c^{-\beta+1}-1}{\beta-1}\left(n/\zeta(\beta)\right)^{1/\beta}=\Theta\left(n^{1/\beta}\right) & \quad \text{if } \beta>1.6.
        \end{cases}\\
        c_1=
        \begin{cases}
            \frac{(1-\beta)^2}{2-\beta} & \quad \text{if } 0<\beta<1,\\
            \frac{1-c}{(\ln 1/c)^2} & \quad \text{if } \beta=1,\\
            \frac{(\beta-1)^2}{2-\beta}\frac{1-c^{2-\beta}}{(c^{-\beta+1}-1)^2} & \quad \text{if } 1<\beta<2\text{ or }\beta>2,\\
            \frac{\ln 1/c}{(1/c-1)^2} & \quad \text{if } \beta=2.
        \end{cases}
    \end{gather}
    In all cases the average degree $d$ and the expansion $\gamma$ are as follows:
    \begin{gather}
        d=c_1n'\\
        \gamma=\frac{d}{2}-\delta
    \end{gather}
\end{theorem}

The proof of the theorem is on page~\pageref*{prf:powerlaw-coin-toss-edge-expansion}.

\begin{lemma}
    \label{lem:coin-toss-lemma-d}
    $\forall\delta>0\;\exists\gamma:$
    If the expected average degree of a graph $H$ on $n'$ vertices is $d>10\ln n'$,
    then $H$ is $(n'/2,\gamma)$ edge expander w.h.p., $\gamma=d/2-\delta$.
\end{lemma}

\begin{proof}
    Given an arbitrary cut $(S,T)$: $e(S,T)=\{(u,v)\in E_H\;|\;u\in S,v\in T\}$.
    Using~\eqref{eq:powerlaw-coin-toss-edge-pr}:
    \begin{equation}
        \E_G[|e(S,T)|]=\frac{\Vol(S)\,\Vol(T)}{\Vol(H)}
        % arg max(x(1-x)) = 0.5
    \end{equation}
    
    The volume of an arbitrary subset $S\subset V_H$ of a fixed size $|S|=s\leq n'/2$
    is $\Vol(S)=sd$, where $d=\Vol(H)/n'$ is the expected average degree of $H$.
    
    Define a random variable $X_S$ to be the size of a cut $(S,V_H\backslash S)$ with expected value $\mu$:
    \begin{gather}
        X_S=|e(S,V_H\backslash S)|\\
        \mu=\E_G[X_S]=\frac{\Vol(S)\Vol(V_H\backslash S)}{\Vol(H)}=\Vol(S)\left(1-\frac{s}{n'}\right)\\
        \frac{\mu}{s}=d\left(1-\frac{s}{n'}\right)\\
        \frac{d}{2}\leq\frac{\mu}{s}\leq d
    \end{gather}
    
    We will use the Chernoff bound for the lower tail~\eqref{eq:chernoff-lower-tail}
    for some $0<\lambda<1$:
    \begin{gather*}
        \gamma s=(1-\lambda)\mu\\
        0<\lambda=1-\gamma s/\mu
    \end{gather*}
    \begin{equation}
        \label{eq:powerlaw-coin-toss-expansion-bound}
        \gamma<\mu/s
    \end{equation}
    
    Therefore, the only requirement is $\gamma<d/2$. Now, as for the expansion of $H$:
    
    $\Pr_{G,S}[S\text{ is non-expanding}]=\Pr_{G,S}[X_S\leq\gamma s]\leq \exp(-\lambda^2\mu/2)
    =\exp(-(\mu-\gamma s)^2/2\mu)=$
    
    $\qquad=\exp(-\mu/2+\gamma s-\gamma^2s^2/2\mu)\leq\exp(\gamma s-\mu/2)$.
    
    $\Pr_G[H\text{ is not }(n'/2,\gamma)\text{ edge expander}]$
    $\leq\sum_{s=1}^{n'/2}{\binom{n'}{s}\Pr_{G,S}[S\text{ is non-expanding}]}\leq$
    
    $\qquad\leq\sum_{s=1}^{n'/2}{\left(\frac{en'}{s}\exp\left(-\frac{(\mu/s-\gamma)^2}{2\mu/s}\right)\right)^s}
    =\sum_{s=1}^{n'/2}{\exp\left(\left(1+\ln\frac{n'}{s}-\frac{(\mu/s-\gamma)^2}{2\mu/s}\right)s\right)}\leq$
    
    $\qquad\leq\sum_{s=1}^{n'/2}{\exp\left(\left(
        1
        -\ln\frac{s}{n'}
        -\frac{d}{2}\left(1-s/n'\right)
        +\gamma
        -\frac{\gamma^2}{2d\left(1-s/n'\right)}
    \right)s\right)}\leq$
    
    $\qquad\leq\sum_{s=1}^{n'/2}{\exp\left(\left(
        1+\ln n'
        -\frac{d}{2}(1-1/2)
        +\gamma
    \right)s\right)}\leq$
    
    $\qquad\leq\sum_{s=1}^{n'/2}{\left(\frac{1}{n'^{(1+\sigma)}}\right)^s}\leq\frac{n'/2}{n'^{(1+\sigma)}}=o(1).$
    
    The last line is obtained if we can satisfy the following inequality
    for some small constant $\sigma>0$:
    \begin{gather*}
        1+\ln n'-\frac{d}{4}+\gamma\leq-(1+\sigma)\ln n'\\
        0<\gamma\leq\frac{d}{4}-(2+\sigma)\ln n'-1\\
        d>4\left((2+\sigma)\ln n'+1\right)\\
        d>10\ln n'
    \end{gather*}
\end{proof}

\begin{lemma}[Size and volume of the subgraph H]
    \label{lem:powerlaw-coin-toss-size-vol-of-h}
    If $d_0=c\,d_{max}$ for $0<c<1$, then:
    \begin{gather}
        n'=
        \begin{cases}
            \left(1-c^{1-\beta}\right)n & \quad \text{if } \beta<1,\\
            \frac{\ln 1/c}{\alpha}n\approx\left(\ln 1/c\right)\frac{n}{\ln n} & \quad \text{if } \beta=1,\\
            \frac{c^{-\beta+1}-1}{\beta-1}\left(n/\zeta(\beta)\right)^{1/\beta} & \quad \text{if } \beta>1. %o(n)
        \end{cases}\\
        \Vol(H)=
        \begin{cases}
            \frac{1-c^{2-\beta}}{2-\beta}e^{2\alpha/\beta} & \quad \text{if } \beta<2,\\
            \left(\ln 1/c\right)e^\alpha & \quad \text{if } \beta=2,\\
            \frac{c^{-\beta+2}-1}{\beta-2}e^{2\alpha/\beta}+\frac{c^{-\beta+1}-1}{2}e^{\alpha/\beta} & \quad \text{if } \beta>2.
        \end{cases}\\
        d=\frac{\Vol(H)}{n'}=\Theta(n')
    \end{gather}
\end{lemma}

\begin{proof}
    Consider~\eqref{eq:powerlaw-coin-toss-n-e} and
    $n'=\sum_{x=d_0}^{d_{max}}\frac{e^\alpha}{x^\beta},
    \;\Vol(H)=\sum_{x=d_0}^{d_{max}}{x\frac{e^\alpha}{x^\beta}}$.
    
    \textbf{Case $\beta<1$}, similarly to~\eqref{eq:powerlaw-coin-toss-vol}:
    \begin{equation*}
        n=\frac{e^{\alpha/\beta}}{1-\beta}
        \qquad n'=\frac{e^\alpha}{1-\beta}\left(d_{max}^{1-\beta}-d_0^{1-\beta}\right)
        =\frac{1-c^{1-\beta}}{1-\beta}e^{\alpha/\beta}
    \end{equation*}
    
    \textbf{Case $\beta=1$:}
    $d_{max}=e^\alpha$.
    \begin{equation*}
        n=\left(\ln d_{max}\right)e^\alpha=\alpha e^\alpha
        \qquad n'=(\ln d_{max}-\ln d_0)e^\alpha=(\ln 1/c)e^\alpha
    \end{equation*}

    \textbf{Case $\beta>1$:}
    
    $n=e^\alpha H_{d_{max},\beta}$
    
    $n'=e^\alpha\left(H_{d_{max},\beta}-H_{c\,d_{max},\beta}\right)
    =\left(1-\frac{H_{c\,d_{max},\beta}}{H_{d_{max},\beta}}\right)n=$
    
    $\qquad=\left(1-\frac{
        \zeta(\beta)-\frac{1}{(\beta-1)(c\,d_{max})^{\beta-1}}-\frac{1}{2(c\,d_{max})^\beta}-O\left(\frac{1}{(c\,d_{max})^{\beta+1}}\right)
    }{
        \zeta(\beta)-\frac{1}{(\beta-1)d_{max}^{\beta-1}}-\frac{1}{2\,d_{max}^\beta}-O\left(\frac{1}{d_{max}^{\beta+1}}\right)
    }\right)n=$

    $\qquad=\left(\frac{
        \frac{c^{-\beta+1}-1}{(\beta-1)d_{max}^{\beta-1}}+\frac{c^{-\beta}-1}{2\,d_{max}^\beta}+O\left(\frac{1}{d_{max}^{\beta+1}}\right)
    }{
        \zeta(\beta)-\frac{1}{(\beta-1)d_{max}^{\beta-1}}-\frac{1}{2\,d_{max}^\beta}-O\left(\frac{1}{d_{max}^{\beta+1}}\right)
    }\right)n=$

    $\qquad=\left(\frac{c^{-\beta+1}-1}{(\beta-1)d_{max}^{\beta-1}}+\frac{c^{-\beta}-1}{2\,d_{max}^\beta}+O\left(\frac{1}{d_{max}^{\beta+1}}\right)\right)e^\alpha=$
    
    $\qquad=\frac{c^{-\beta+1}-1}{\beta-1}d_{max}+\frac{c^{-\beta}-1}{2}+O\left(\frac{1}{d_{max}}\right)\approx$
    
    $\qquad\approx\frac{c^{-\beta+1}-1}{\beta-1}e^{\alpha/\beta}
    \approx\frac{c^{-\beta+1}-1}{\beta-1}\left(\frac{n}{\zeta(\beta)}\right)^{1/\beta}$.
    
    Now for the expected volume, \textbf{case $\beta<2$}:
    \begin{gather*}
        \Vol(H)=\frac{e^\alpha}{2-\beta}\left(d_{max}^{2-\beta}-d_0^{2-\beta}\right)
        =\frac{1-c^{2-\beta}}{2-\beta}e^{2\alpha/\beta}
    \end{gather*}
    
    \textbf{Case $\beta=2$:}
    $d_{max}=e^{\alpha/2}$.
    \begin{gather*}
        \Vol(H)=(\ln d_{max}-\ln d_0)e^\alpha=(\ln 1/c)e^\alpha
    \end{gather*}
    
    \textbf{Case $\beta>2$:}
    \begin{gather*}
        \Vol(H)=e^\alpha\left(H_{d_{max},\beta-1}-H_{c\,d_{max},\beta-1}\right)=\\
        =\left(
            -\frac{1}{(\beta-2)d_{max}^{\beta-2}}-\frac{1}{2d_{max}^{\beta-1}}
            +\frac{1}{(\beta-2)(c\,d_{max})^{\beta-2}}+\frac{1}{2(c\,d_{max})^{\beta-1}}+O\left(\frac{1}{d_{max}^\beta}\right)
        \right)e^\alpha=\\
        =\left(\frac{c^{-\beta+2}-1}{(\beta-2)d_{max}^{\beta-2}}+\frac{c^{-\beta+1}-1}{2d_{max}^{\beta-1}}+O\left(\frac{1}{d_{max}^\beta}\right)\right)e^\alpha=\\
        =\frac{c^{-\beta+2}-1}{\beta-2}d_{max}^2+\frac{c^{-\beta+1}-1}{2}d_{max}+O(1)\approx\\
        \approx\frac{c^{-\beta+2}-1}{\beta-2}e^{2\alpha/\beta}+\frac{c^{-\beta+1}-1}{2}e^{\alpha/\beta}
    \end{gather*}

    Finally, the average degree is always $\Theta(n')$:
    \begin{equation*}
        d=\frac{\Vol(H)}{n'}=
        \begin{cases}
            \frac{1-\beta}{2-\beta}\frac{1-c^{2-\beta}}{1-c^{1-\beta}}e^{\alpha/\beta} & \quad \text{if } \beta<1,\\
            \frac{1-c}{\ln 1/c}e^\alpha & \quad \text{if } \beta=1,\\
            \frac{\beta-1}{2-\beta}\frac{1-c^{2-\beta}}{c^{-\beta+1}-1}e^{\alpha/\beta} & \quad \text{if } 1<\beta<2,\\
            \frac{\ln 1/c}{1/c-1}e^{\alpha/2} & \quad \text{if } \beta=2,\\
            \frac{\beta-1}{\beta-2}\frac{c^{-\beta+2}-1}{c^{-\beta+1}-1}e^{\alpha/\beta}+O(1) & \quad \text{if } \beta>2.
        \end{cases}
    \end{equation*}
\end{proof}

Considering $H$ independently from $G\backslash H$,
we need to account for $e(H,G\backslash H)$:
\begin{gather}
    \Vol(G)=\begin{cases}
        \Theta\left(n^2\right) & \quad \text{if } \beta<1,\\
        \Theta\left(\left(\frac{n}{\log n}\right)^2\right) & \quad \text{if } \beta=1,\\
        \Theta\left(n^{2/\beta}\right) & \quad \text{if } 1<\beta<2,\\
        \Theta\left(n\log n\right) & \quad \text{if } \beta=2,\\
        \Theta\left(n\right) & \quad \text{if } \beta>2.
    \end{cases}\\
    \Vol(H)=\sum_{v\in H}{w_v}=\Theta\left(n'^2\right)=\begin{cases}
        \Theta\left(n^2\right) & \quad \text{if } \beta<1,\\
        \Theta\left(\left(\frac{n}{\log n}\right)^2\right) & \quad \text{if } \beta=1,\\
        \Theta\left(n^{2/\beta}\right) & \quad \text{if } \beta>1.
    \end{cases}\\
    2\,|e(H)|=\Vol(H)-|e(H,G\backslash H)|\\
    |e(H,G\backslash H)|=\frac{\Vol(H)\Vol(G\backslash H)}{\Vol(G)}\\
    |e(H)|=\frac{\Vol(H)^2}{2\Vol(G)}
\end{gather}
\begin{equation}
    \begin{split}
        |e(G)|=&|e(H)|+|e(H,G\backslash H)|+|e(G\backslash H)|=\\
        =&\frac{\Vol(H)^2/2+\Vol(H)\Vol(G\backslash H)+\Vol(G\backslash H)^2/2}{\Vol(G)}=\\
        =&\frac{(\Vol(H)+\Vol(G\backslash H))^2}{2\Vol(G)}
        =\frac{\Vol(G)^2}{2\Vol(G)}=\frac{\Vol(G)}{2}
    \end{split}
\end{equation}

How much smaller would $2\,e(H)$ be than $\Vol(H)$ that we use?
\begin{gather}
    2\,e(H)=\frac{\Vol(H)^2}{\Vol(G)}=\Vol(H)\,x\\
    x=\frac{\Vol(H)}{\Vol(G)}=\begin{cases}
        \Theta(1) & \quad \text{if } \beta<2,\\
        \frac{1}{\log n} & \quad \text{if } \beta=2,\\
        \frac{1}{n^{1-2/\beta}} & \quad \text{if } \beta>2.
    \end{cases}
\end{gather}
It means the larger $\beta$, the fewer edges are between high-degree vertices in $H$
than edges from $H$ to low-degree vertices in $G\backslash H$.

\begin{lemma}[Linear size subgraph H]
    \label{lem:powerlaw-coin-toss-size-vol-of-h-large}
    In order to get $n'=\Theta(n)$, we need to include more vertices in $H$:
    \begin{itemize}
        \item $\beta=1$: if $d_0=d_{max}/\sqrt{n}$, then $n'=n/2$;
        \item $1<\beta\leq 1.6$: if $d_0=\frac{d_{max}}{n^{1/\beta}}$, then $n'=\Theta(n)$.
    \end{itemize}
\end{lemma}

\begin{proof}
    \textbf{Case $\beta=1$:}
    $d_{max}=e^\alpha$.
    
    %todo temp
    \begin{gather*}
        n=\alpha e^\alpha\qquad\alpha\approx\ln n\\
        d_0=d_{max}e^{-x}\\
        \ln d_0=\ln d_{max}-x\\
        n'=(\ln d_{max}-\ln d_0)\frac{n}{\alpha}=\frac{xn}{\alpha}
    \end{gather*}

    $x=\frac{\ln n}{2}:\quad d_0=\frac{d_{max}}{\sqrt{n}},\quad n'\approx\frac{n}{2}$
    
    $x=c\;\ln n:\quad d_0=\frac{d_{max}}{n^c},\quad n'\approx cn$
    
    \begin{gather*}
        \Vol(H)=e^\alpha(d_{max}-d_{max}n^{-c})=e^{2\alpha}(1-n^{-c})\\
        d=\frac{\Vol(H)}{n'}=\frac{e^\alpha}{c\alpha}(1-n^{-c}) %todo same c (for vol & d_0)?
        =\frac{n'}{c^2\alpha^2}(1-(n'/c)^{-c})
        \approx\frac{1}{c^2}(1-(n'/c)^-c)\frac{n'}{(\ln (n'/c))^2}
    \end{gather*}
    
    If $c=1/2,\quad d\approx4\left(1-\frac{1}{\sqrt{2n'}}\right)\frac{n'}{(\ln (2n'))^2}=\omega(\ln n')$.
    
    \textbf{Case $\beta>1$:}
    \begin{gather*}
        n'=\left(1-\frac{H_{x,\beta}}{H_{d_{max},\beta}}\right)n\\
        1-\frac{H_{x,\beta}}{H_{d_{max},\beta}}=c\\
        (1-c)H_{d_{max},\beta}=H_{x,\beta}\\
        (1-c)\sum_{k=1}^{d_{max}}\frac{1}{k^\beta}=\sum_{k=1}^{x}\frac{1}{k^\beta}
        \qquad(1-c)\sum_{k=x+1}^{d_{max}}\frac{1}{k^\beta}=c\sum_{k=1}^{x}\frac{1}{k^\beta}
    \end{gather*}
    % todo
    % syms c k d x
    % X=solve((1-c)*symsum(1/k,k,1,d)-symsum(1/k,k,1,x), x)
    % > Warning: unable to find explicit solution.
    
    Known special case:
    $H_{2x,2}=\frac{1}{2}\left(\zeta(2)+\frac{1}{2}\left(H_{x,2}+H_{x-\frac{1}{2},2}\right)\right)$
    
    \begin{gather*}
        (1-c)\left(\zeta(\beta)-\frac{1}{(\beta-1)d_{max}^{\beta-1}}-\frac{1}{2\,d_{max}^\beta}-O\left(\frac{1}{d_{max}^{\beta+1}}\right)\right)=\\
        =\zeta(\beta)-\frac{1}{(\beta-1)x^{\beta-1}}-\frac{1}{2\,x^\beta}-O\left(\frac{1}{x^{\beta+1}}\right)\\
        \frac{1}{(\beta-1)x^{\beta-1}}+\frac{1}{2\,x^\beta}-\frac{1-c}{(\beta-1)d_{max}^{\beta-1}}-\frac{1-c}{2\,d_{max}^\beta}+O\left(\frac{1}{x^{\beta+1}}\right)=c\;\zeta(\beta)
    \end{gather*}
    
    \begin{gather*}
        (1-c)H_{d_{max},\beta}=H_{x,\beta}\\
        (1-c)\sum_{k=1}^{d_{max}}\frac{1}{k^\beta}=\sum_{k=1}^{x}\frac{1}{k^\beta}\\
        \zeta(\beta)-H_{a,\beta}=\int_a^\infty{\frac{1}{t^\beta}\,dt}
        +\frac{1}{2a^\beta}
        -\sum_{i=2}^k{\frac{b_i}{i!}f^{(i-1)}(a)}
        -\int_a^\infty{\frac{B_k(\{1-t\})}{k!}f^{(k)}(t)\,dt}=\\
        =\frac{1}{(\beta-1)a^{\beta-1}}
        +\frac{1}{2a^\beta}
        -\frac{-\beta}{12a^{\beta+1}}
        -O\left(\frac{1}{a^{\beta+1}}\right)\\\\
        (1-c)H_{d_{max},\beta}=(1-c)\left(\zeta(\beta)-\frac{1}{(\beta-1)d_{max}^{\beta-1}}
        -\frac{1}{2d_{max}^\beta}
        +O\left(\frac{1}{d_{max}^{\beta+1}}\right)\right)\\
        H_{d_0,\beta}=\zeta(\beta)-\frac{1}{(\beta-1)d_0^{\beta-1}}
        -\frac{1}{2d_0^\beta}
        +O\left(\frac{1}{d_0^{\beta+1}}\right)
        % syms c z d b s
        % X=solve((1-c)*(z-1/((b-1)*d^(b-1))-1/(2*d^b))-(z-1/((b-1)*(s*d)^(b-1))-1/(2*(s*d)^b)), s)
        % syms a b c d z
        % X=solve((1/(a^(b-1))-(1-c))/((b-1)*d^(b-1))+(1/(a^b)-(1-c))/(2*d^b)-c*z, a)
    \end{gather*}
    If $d_0=xd_{max}$:
    \begin{gather*}
        d_{max}=e^{\alpha/\beta}\\
        n=e^\alpha H_{d_{max},\beta}\approx e^\alpha\zeta(\beta)
        =d_{max}^\beta\zeta(\beta)\\\\
        n'=e^\alpha\left(H_{d_{max},\beta}-H_{x\,d_{max},\beta}\right)=\\
        =e^\alpha\left(
            \frac{x^{-\beta+1}-1}{(\beta-1)d_{max}^{\beta-1}}
            +\frac{x^{-\beta}-1}{2d_{max}^\beta}
            +O\left(\frac{1}{d_{max}^{\beta+1}}\right)
        \right)=\\
        =\frac{x^{-\beta+1}-1}{(\beta-1)}d_{max}
        +\frac{x^{-\beta}-1}{2}
        +O\left(\frac{1}{d_{max}}\right)
        \approx\frac{x^{-\beta+1}-1}{(\beta-1)}\left(\frac{n}{\zeta(\beta)}\right)^{1/\beta}
    \end{gather*}
    We now choose $x=n^y$:
    \begin{gather*}
        n'\approx\frac{n^{1/\beta-y(\beta-1)}}{(\beta-1)\,\zeta(\beta)^{1/\beta}}\\
        1/\beta-y(\beta-1)=1\\
        y=\frac{1/\beta-1}{\beta-1}=-1/\beta
    \end{gather*}
    $d_0=\frac{d_{max}}{n^{1/\beta}}\implies
    n'\approx\frac{n}{(\beta-1)\,\zeta(\beta)^{1/\beta}}=\Theta(n)
    \qquad\qquad\qquad\qquad\sum_{i=1}^{\infty}{\frac{1}{i^2}}=\frac{\pi^2}{6}$
        
    We note that $(\beta-1)\,\zeta(\beta)^{1/\beta}>1$ requires $\beta\leq 1.6$.

    Finally, we get the expected average degree $d$ for different ranges of $\beta$:
    \begin{align*}
        d&=\frac{\Vol(H)}{n'}=\frac{H_{d_{max},\beta-1}-H_{d_0,\beta-1}}{H_{d_{max},\beta}-H_{d_0,\beta}}\\
        1<\beta<2:\quad d&=\frac{d_{max}^{2-\beta}-d_0^{2-\beta}}{(2-\beta)\,\Theta(1)}
        =\frac{1-n^{-(2-\beta)/\beta}}{(2-\beta)\,\Theta(1)}d_{max}^{2-\beta}=\Theta\left(d_{max}^{2-\beta}\right)\\
        \beta=2:\quad d&=\frac{\ln d_{max}-\ln d_0}{\Theta(1)}=\frac{\frac{1}{\beta}\ln n}{\Theta(1)}\\
        \beta>2:\quad d&=\Theta(1)
    \end{align*}
    
    We can see that~\autoref{lem:coin-toss-lemma-d} is applicable for
    $1\leq\beta\leq 1.6$ because $d=\omega(\ln n')$.
\end{proof}

\begin{proof}[Proof of~\autoref{thm:powerlaw-coin-toss-edge-expansion}]
    \label{prf:powerlaw-coin-toss-edge-expansion}
    For $1\leq\beta\leq 1.6$ we use
    \autoref{lem:powerlaw-coin-toss-size-vol-of-h-large}
    and~\autoref{lem:coin-toss-lemma-d} and we are done.
    
    Otherwise, we get $n'$ and $\Vol(H)$ from~\autoref{lem:powerlaw-coin-toss-size-vol-of-h},
    which covers all possible values of $\beta$.
    
    \textbf{Case $0<\beta<1$:}
    \begin{equation*}
        \Vol(S)=s\frac{1-\beta}{2-\beta}\frac{1-c^{2-\beta}}{1-c^{1-\beta}}e^{\alpha/\beta}
        =\frac{(1-\beta)^2}{2-\beta}\frac{1-c^{2-\beta}}{(1-c^{1-\beta})^2}sn'
    \end{equation*}
    
    Let $c_1=\frac{(1-\beta)^2}{2-\beta}\frac{1-c^{2-\beta}}{(1-c^{1-\beta})^2}>0$,
    then $\Vol(S)=c_1sn'$.
    
    $\Pr_{G,S}[S\text{ is non-expanding}]\leq\exp(\gamma s-\mu/2)
    =exp\left(\left(\gamma-c_1(n'-s)/2\right)s\right)$.
    
    $\Pr_G[H\text{ is not }(\epsilon n',\gamma)\text{ edge expander}]
    \leq\sum_{s=1}^{\epsilon n'}{\binom{n'}{s}\Pr_{G,S}[S\text{ is non-expanding}]}\leq$
    
    $\qquad\leq\sum_{s=1}^{\epsilon n'}{\left(e^{\gamma+1}\frac{n'}{s}e^{-c_1(n'-s)/2}\right)^s}=o(1)$.
    
    When $1\leq s\leq\epsilon n'$, the last equality holds for any $\epsilon>0$ satisfying the following condition:
    \begin{gather*}
        e^{\gamma+1}\frac{n'}{s}e^{-c_1(n'-s)/2}<\frac{1}{10}\\
        e^{\gamma+1}n'e^{-c_1n'(1-\epsilon)/2}<\frac{1}{10}\\
        c_1n'(1-\epsilon)/2>\ln\left(10e^{\gamma+1}n'\right)\\
        \epsilon<1-\frac{2}{c_1n'}\ln\left(10e^{\gamma+1}n'\right)
        =1-\frac{\ln(c_2n')}{c_3n'}=1-o(1)
    \end{gather*}
    
    We choose $\epsilon=1/2$.
    
    To satisfy the requirement~\eqref{eq:powerlaw-coin-toss-expansion-bound} when $1\leq s\leq n'/2$:
    \begin{gather*}
        \gamma<\mu/s=c_1(1-\frac{s}{n'})n'\\
        \gamma=\frac{c_1}{2}n'-\delta\text{, for any }\delta\geq 0
    \end{gather*}
    
    Finally, we restrict $c<1/3$ to have $c_1\leq1$ and $\Vol(S)\leq sn'$.
    
    Note that if $0<\beta<1$ and $d_0=1$, then $n'=n$ and $\Vol(H)=\Vol(G)$ but the argument still holds,
    i.e., the whole graph $G$ is an edge expander w.h.p.
    
    \textbf{Case $\beta=1$:}
    \begin{equation*}
        \Vol(S)=s\frac{1-c}{\ln 1/c}e^\alpha=\frac{1-c}{(\ln 1/c)^2}sn'
    \end{equation*}
    
    Let $c_1=\frac{1-c}{(\ln 1/c)^2}>0$, then $\Vol(S)=c_1sn'$.
    
    The rest of the proof is identical to the previous case, but necessarily $c>0$.
    
    \textbf{Case $1<\beta<2$:}
    \begin{equation*}
        \Vol(S)=s\frac{\beta-1}{2-\beta}\frac{1-c^{2-\beta}}{c^{-\beta+1}-1}e^{\alpha/\beta}
        =\frac{(\beta-1)^2}{2-\beta}\frac{1-c^{2-\beta}}{(c^{-\beta+1}-1)^2}sn'
    \end{equation*}
    
    Let $c_1=\frac{(\beta-1)^2}{2-\beta}\frac{1-c^{2-\beta}}{(c^{-\beta+1}-1)^2}>0$,
    then $\Vol(S)=c_1sn'$.
    
    The case is identical to $\beta=1$.
    
    \textbf{Case $\beta=2$:}
    \begin{equation*}
        \Vol(S)=s\frac{\ln 1/c}{1/c-1}e^{\alpha/2}
        =\frac{\ln 1/c}{(1/c-1)^2}sn'
    \end{equation*}
    
    Analogously, $c_1=\frac{\ln 1/c}{(1/c-1)^2}>0$ and again $\Vol(S)=c_1sn'$.
    
    \textbf{Case $\beta>2$:}
    similarly to the case $1<\beta<2$,
    \begin{equation*}
        \Vol(S)=s\left(\frac{\beta-1}{\beta-2}\frac{c^{-\beta+2}-1}{c^{-\beta+1}-1}e^{\alpha/\beta}+O(1)\right)
        \approx\frac{(\beta-1)^2}{\beta-2}\frac{c^{-\beta+2}-1}{(c^{-\beta+1}-1)^2}sn'
    \end{equation*}

%    \textbf{Special case $1<\beta<2$ and $c=0$:} here $H=G$.
%    \begin{gather*}
%        n'=n=\zeta(\beta)e^\alpha\\
%        \Vol(H)=\Vol(G)=\frac{e^{2\alpha/\beta}}{2-\beta}=\frac{1}{2-\beta}\left(\frac{n}{\zeta(\beta)}\right)^{2/\beta}\\
%        \Vol(S)=s\frac{\Vol(G)}{n}=\frac{sn^{2/\beta-1}}{(2-\beta)\zeta(\beta)^{2/\beta}}
%    \end{gather*}
%    
%    Let $c_1=\frac{1}{(2-\beta)\zeta(\beta)^{2/\beta}}>0$, then $\Vol(S)=c_1sn^{2/\beta-1}$.
%    
%    $\Pr_G[G\text{ is not }(\epsilon n,\gamma)\text{ edge expander}]
%    \leq\sum_{s=1}^{\epsilon n}{\binom{n}{s}\exp(\gamma s-\mu/2)}\leq$
%    
%    $\qquad\leq\sum_{s=1}^{\epsilon n}{\binom{n}{s}\exp\left(\left(\gamma-c_1n^{2/\beta-1}(1-s/n)/2\right)s\right)}\leq$
%    
%    $\qquad\leq\sum_{s=1}^{\epsilon n}{\left(
%        e^{\gamma+1}
%        \frac{n}{s}
%        e^{-c_1n^{2/\beta-1}(1-s/n)/2}
%    \right)^s}\leq\ldots$
%
%    $0<2/\beta-1<1$, $(1-\epsilon)\leq(1-s/n)<1$
\end{proof}

%\textbf{Special case} when $\beta=3$:
%\begin{gather*}
%    n'=\frac{e^\alpha}{2}\left(\frac{1}{d_0^2}-\frac{1}{d_{max}^2}\right)
%    \qquad \Vol(H)=e^\alpha\left(\frac{1}{d_0}-\frac{1}{d_{max}}\right)
%    =e^\alpha\frac{d_{max}-d_0}{d_{max}d_0}\\
%    \Vol(S)=2s\frac{d_{max}d_0}{d_{max}+d_0}\\
%    \mu=\Vol(S)\left(1-\frac{2s}{e^\alpha}\frac{d_{max}^2d_0^2}{d_{max}^2-d_0^2}\right)
%\end{gather*}
%
%Further, if $d_0=d_{max}/100$:
%\begin{gather*}
%    n'=\frac{e^\alpha}{2\,d_{max}^2}(100^2-1)
%    \qquad \Vol(H)=e^\alpha\frac{99}{d_{max}}\\
%    \Vol(S)=\frac{2s}{101}d_{max}\\
%    \mu=\Vol(S)\left(1-\frac{2s}{e^\alpha}\frac{d_{max}^2}{100^2-1}\right)\\
%\end{gather*}
%
%$\Pr_{G,S}[S\text{ is non-expanding}]\leq \exp(-\left(\mu-\gamma s\right)^2/2\mu)
%=\exp(-\mu/2+\gamma s-\gamma^2s^2/2\mu)$
%
%$\qquad\leq \exp(\gamma s-\mu/2)
%=exp\left(\left(\gamma-\left(1-\frac{2s}{e^\alpha}\frac{d_{max}^2}{100^2-1}\right)d_{max}/101\right)s\right)$
%
%$\Pr_G[H\text{ is not }(\epsilon n',\gamma)\text{ edge expander}]$
%
%$\qquad\leq\sum_{s=1}^{\epsilon n'}\left(\frac{n'}{s}\,exp\left(
%    \gamma-
%    \left(1-\frac{2s}{e^\alpha}\frac{d_{max}^2}{100^2-1}\right)d_{max}/101
%    +1
%\right)\right)^s$.
%
%Remember that $d_{max}=e^{\alpha/3}$, therefore:
%\begin{gather*}
%    n'=\frac{100^2-1}{2}e^{\alpha/3}
%    \qquad \Vol(H)=99\,e^{(2/3)\alpha}\\
%    \Vol(S)=\frac{2s}{101}e^{\alpha/3}\\
%    \mu=\frac{2s}{101}e^{\alpha/3}\left(1-\frac{2s}{100^2-1}e^{-\alpha/3}\right)
%    =\frac{2s}{101}\left(e^{\alpha/3}-\frac{2s}{100^2-1}\right)
%    =\frac{2s}{101}e^{\alpha/3}\left(1-\frac{s}{n'}\right)
%\end{gather*}
%
%$\Pr_{G,S}[S\text{ is non-expanding}]\leq \exp(\gamma s-\mu/2)
%=exp\left(\left(\gamma-\left(1-\frac{s}{n'}\right)\frac{e^{\alpha/3}}{101}\right)s\right)$
%
%$\Pr_G[H\text{ is not }(\epsilon n',\gamma)\text{ edge expander}]
%\leq\sum_{s=1}^{\epsilon n'}{\binom{n'}{s}\Pr_{G,S}[S\text{ is non-expanding}]}$
%
%$\leq\sum_{s=1}^{\epsilon n'}{
%    \left(\frac{en'}{s}\right)^s
%    exp\left(\left(\gamma-\frac{2}{101(100^2-1)}(n'-s)\right)s\right)
%}$
%
%$=\sum_{s=1}^{\epsilon n'}{\left(
%    e^{\gamma+1}
%    \frac{n'}{s}
%    exp\left(-\frac{2}{101(100^2-1)}(n'-s)\right)
%\right)^s}
%=\sum_{s=1}^{\epsilon n'}{\left(
%    c_1
%    \frac{n'}{s}
%    e^{-c_2(n'-s)}
%\right)^s}=o(1)$.
%
%Moreover, \eqref{eq:powerlaw-coin-toss-expansion-bound} is satisfied
%whenever $\gamma<\frac{2}{101}d_{max}$.

\section{Permutation Model}

According to our choice of $H$, $n'=|V_H|$ is the largest number from $\{1,\ldots,n\}$ s.t.:
\begin{gather}
    \deg(n')=\frac{pn}{n'^\beta}\geq d_0
    \qquad n'\approx(pn/d_0)^{1/\beta}\\
    \vol(H)=\sum_{v=1}^{n'}{\frac{pn}{v^\beta}}=pnH_{n',\beta}=d_0n'^\beta H_{n',\beta}\\
    \vol(S)=s\frac{\vol(H)}{n'}
\end{gather}

We can apply~\autoref{thm:gms} to show small constant edge expansion of $H$,
as long as $d_0\geq 3$. For larger expansion we have the following theorem.

\begin{theorem}[Existence of an edge expander]
    \label{thm:powerlaw-permutation-edge-expansion}
    $\forall 0<p<1\;\exists d_0=d_0(p,n,\beta)\;\forall\delta>0\;\exists\gamma,\epsilon>0:$
    let $G$ be a random power-law graph where each vertex $v$ has $\deg(v)=pnv^{-\beta}$,
    and $H$ is its induced subgraph of size $n'$ obtained by retaining vertices of degree at least $d_0$.
    
    Then if $\beta=0$, the whole graph $G$ is $(\epsilon n,\gamma)$ edge expander w.h.p.
    
    Otherwise, $H$ is $(\epsilon n',\gamma)$ edge expander w.h.p.
    
    When $\beta>1$, the additional requirement is $p\,\zeta(\beta)>2$.
    More roughly, $\zeta(\beta)>2$ or $\beta<1.72$,
    and the smaller $p$, the smaller $\beta$ for which subsets still expand.
    \begin{gather}
        n'=
        \begin{cases}
            n & \quad \text{if } \beta=0,\\
            n/2 & \quad \text{if } \beta>0.
        \end{cases}\\
        d_0=
        \begin{cases}
            pn & \quad \text{if } \beta=0,\\
            2^\beta pn^{1-\beta} & \quad \text{if } \beta>0.
        \end{cases}\\
        d=
        \begin{cases}
            d_0\\
            d_0/(1-\beta)\\
            d_0\ln n'\\
            d_0\,\zeta(\beta)\,n'^{\beta-1}
        \end{cases}=
        \begin{cases}
            pn & \quad \text{if } \beta=0,\\
            \frac{2^\beta p}{1-\beta}n^{1-\beta} & \quad \text{if } 0<\beta<1,\\
            2p\ln(n/2) & \quad \text{if } \beta=1,\\
            2p\,\zeta(\beta) & \quad \text{if } \beta>1.
        \end{cases}\\
        \gamma=\begin{cases}
            d-2-\delta & \quad \text{if } \beta=0,\\
            d/2 & \quad \text{if } \beta>0.
        \end{cases}
    \end{gather}
\end{theorem}

\begin{theorem}[Existence of a vertex expander]
    \label{thm:powerlaw-permutation-vertex-expansion}
    $\forall 0<p<1\;\exists d_0=d_0(p,n,\beta)\;\forall\delta>0\;\exists\gamma,\epsilon>0:$
    let $G$ be a random power-law graph where each vertex $v$ has $\deg(v)=pnv^{-\beta}$,
    and $H$ is its induced subgraph of size $n'$ obtained by retaining vertices of degree at least $d_0$.
    
    Then if $\beta=0$, the whole graph $G$ is $(\epsilon n,\gamma)$ vertex expander w.h.p.
    
    Otherwise, if $0<\beta<1$, $H$ is $(\epsilon n',\gamma)$ vertex expander w.h.p.
    At the same time, either $\epsilon=1/2$ and $\gamma=1-\delta$,
    or $\epsilon$ is sufficiently small and
    \begin{equation}
        \gamma=\begin{cases}
            d/2-2-\delta & \quad \text{if } \beta=0,\\
            d_0/2-2-\delta & \quad \text{if } 0<\beta<1.
        \end{cases}
    \end{equation}
\end{theorem}

The proof of this theorem can be found on page~\pageref*{prf:powerlaw-permutation-vertex-expansion}.

\begin{proof}[Proof of~\autoref{thm:powerlaw-permutation-edge-expansion}]
    \textbf{Case $\beta=0$:}
    $\deg(v)=pn$ for each $v\in V$, so we choose $d_0=pn$, then trivially $H=G$ and $n'=n$.
    This reduces the proof to known case of regular graphs described in~\autoref{subsec:edge-expansion-reg}.
    
    As for the other cases when $\beta>0$, we want our subgraph $H$ to have linear size, so we need to choose appropriate $d_0$:
    \begin{gather*}
        n'=(pn/d_0)^{1/\beta}=cn\\
        d_0=\frac{pn}{(cn)^\beta}=\frac{p}{c^\beta}n^{1-\beta}
    \end{gather*}
    Fixing $c=1/2$ gives:
    \begin{gather*}
        n'=n/2\\
        d_0=2^\beta pn^{1-\beta}=2pn'^{1-\beta}
    \end{gather*}
    
    $\Pr_G[H\text{ is not }(\epsilon n',\gamma)\text{ edge expander}]
    =\sum_{s=1}^{\epsilon n'}{
        \binom{n'}{s}
        \Pr_{G,S}[|e(S,V_H\backslash S)| < \gamma s]
    }=$
    
    $\qquad=\sum_{s=1}^{\epsilon n'}{
        \binom{n'}{s}
        \Pr_{G,S}\left[|e(S,S)|\geq\frac{\vol(S)-\gamma s}{2}\right]
    }\leq$
    
    $\qquad\leq\sum_{s=1}^{\epsilon n'}{
        \binom{n'}{s}
        \frac{\binom{\vol(H)/2}{(\vol(S)-\gamma s)/2}\binom{\vol(H)-(\vol(S)-\gamma s)}{\gamma s}}{\binom{\vol(H)}{\vol(S)}}
    }\leq$
    
    $\qquad\leq\sum_{s=1}^{\epsilon n'}{
        \left(\frac{en'}{s}\right)^s
        \frac{
            \left(e\frac{\vol(H)}{\vol(S)-\gamma s}\right)^{(\vol(S)-\gamma s)/2}
            \left(e\frac{\vol(H)-(\vol(S)-\gamma s)}{\gamma s}\right)^{\gamma s}
        }
        {\left(\frac{\vol(H)}{\vol(S)}\right)^{\vol(S)}}
    }\leq$
    
    $\qquad\leq\sum_{s=1}^{\epsilon n'}{\left(
        \frac{en'}{s}
        \frac{
            \left(\frac{e\vol(H)}{(\vol(H)/n'-\gamma)s}\right)^{(\vol(H)/n'-\gamma)/2}
            \left(\frac{e\vol(H)}{\gamma s}\right)^{\gamma}
        }
        {\left(\frac{n'}{s}\right)^{\vol(H)/n'}}
    \right)^s}=$

    $\qquad=\sum_{s=1}^{\epsilon n'}{\left(
        \frac{e^{1+\gamma/2}}{\gamma^\gamma}
        e^{d/2}
        \left(\frac{1}{d-\gamma}\right)^{(d-\gamma)/2}
        \left(\frac{\vol(H)}{s}\right)^{(d+\gamma)/2}
        \left(\frac{s}{n'}\right)^{d-1}
    \right)^s}=$

    $\qquad=\sum_{s=1}^{\epsilon n'}{\left(
    \frac{e^{1+\gamma/2}}{\gamma^\gamma}
    e^{d/2}
    \frac{d^{(d+\gamma)/2}}{(d-\gamma)^{(d-\gamma)/2}}
    \left(\frac{s}{n'}\right)^{d/2-\gamma/2-1}
    \right)^s}$,
    
    where $d=\vol(H)/n'$ is the average degree of $H$.
    
    Unfortunately, we cannot use~\autoref{pro:bound-prob-large-sets}
    because of the condition $c_1=o\left(n'^{c_2-1}\right)$.
    
    In order to upper bound the probability of $H$ not being an edge expander we need
    \begin{equation}
        \label{eq:powerlaw-permutation-d-req}
        d-\gamma-2>0
    \end{equation}
    
    Let $\gamma=d/2$ so we can simplify further.
    Note that~\eqref{eq:powerlaw-permutation-d-req} becomes $d>4$.
    
    $\Pr_G[H\text{ is not }(\epsilon n',\gamma)\text{ edge expander}]\leq\sum_{s=1}^{\epsilon n'}{\left(
        \frac{e^{1+d/4}}{(d/2)^{d/2}}
        e^{d/2}
        \frac{d^{3d/4}}{(d/2)^{d/4}}
        \left(\frac{s}{n'}\right)^{d/4-1}
    \right)^s}=$
    
    $\qquad=\sum_{s=1}^{\epsilon n'}{\left(
        e
        (2e)^{3d/4}
        \left(\frac{s}{n'}\right)^{d/4-1}
    \right)^s}$

    What's left to argue is that in all cases this expression may be upper bounded by $o(1)$, while also maintaining $d>4$.
    
    \textbf{Case $0<\beta<1$:}
    \begin{gather*}
        \vol(H)=\frac{d_0}{1-\beta}n'\\
        d=\frac{d_0}{1-\beta}=\frac{2p}{1-\beta}n'^{1-\beta}=\Theta(n'^{1-\beta})
    \end{gather*}
    
    Clearly, our choice of $d_0$ satisfies~\eqref{eq:powerlaw-permutation-d-req}.

    As for the summation, we consider small and large values of $s$ separately.
    \begin{itemize}
        \item $s\leq\sqrt{n'}:
        \sum_{s=1}^{\sqrt{n'}}{\left(
            e
            (2e)^{3d/4}
            \left(\frac{s}{n'}\right)^{d/4-1}
        \right)^s}
        \leq e^{\sqrt{n'}}(2e)^{\sqrt{n'}3d/4}\sum_{s=1}^{\sqrt{n'}}{
            \left(\frac{s}{n'}\right)^{(d/4-1)s}
        }$
    
        $\qquad\leq e^{\sqrt{n'}}(2e)^{\sqrt{n'}3d/4}\sum_{s=1}^{\sqrt{n'}}{
            \left(\frac{1}{\sqrt{n'}}\right)^{d/4-1}
        }$

        $\qquad\leq e^{\sqrt{n'}}(2e)^{\sqrt{n'}3d/4}
        \frac{\sqrt{n'}}{n'^{d/4-1}}$
        
        $\qquad=e^{n'^{1/2}}(2e)^{\frac{3}{4}\Theta(n'^{3/2-\beta})}
        \frac{1}{n'^{\Theta(n'^{1-\beta})/4-3/2}}=o(1)$

        \item otherwise:
        $\sum_{s=\sqrt{n'}+1}^{\epsilon n'}{\left(
            e
            (2e)^{3d/4}
            \left(\frac{s}{n'}\right)^{d/4-1}
        \right)^s}
        <\sum_{s=\sqrt{n'}+1}^{\epsilon n'}{10^{-s}}\leq\frac{\epsilon n'}{10^{\sqrt{n'}}}=o(1)$
        
        For this we need:
        $e(2e)^{3d/4}\epsilon^{d/4-1}<\frac{1}{10}$, or
        $\frac{((2e)^3\epsilon)^{d/4}}{\epsilon}<\frac{1}{10e}$.
        
        We can choose $\epsilon=\frac{1}{10(2e)^3}$, and we are done because $d=\Theta(n'^{1-\beta})$.
    \end{itemize}
    
    $\Pr_G[H\text{ is not }(\epsilon n',\gamma)\text{ edge expander}]\leq o(1)$.
    
    \textbf{Case $\beta=1$:}
    \begin{gather*}
        n'=pn/d_0\\
        \vol(H)=d_0n'\ln n'\\
        d=\frac{\vol(H)}{n'}=d_0\ln n'=\Theta(n'^{1-\beta}\ln n')
    \end{gather*}
    
    This is identical to the previous case $0<\beta<1$.
    
    \textbf{Case $\beta>1$:}
    \begin{gather*}
        \vol(H)=\zeta(\beta)\,d_0n'^\beta\\
        d_0=2pn'^{1-\beta}\\
        d=\zeta(\beta)\,d_0n'^{\beta-1}=2p\,\zeta(\beta)=O(1)
    \end{gather*}
    Here we get an additional requirement $p\,\zeta(\beta)>2$, because we need $d>4$.
    As~$0<p<1$ and $\zeta(\beta)$ is monotonically decreasing, it serves as an upper bound for $\beta$.

    Finally, let $c_1=e(2e)^{3d/4}$, $c_2=d/4-1$ be some positive constants,
    then for sufficiently small $\epsilon$:

    $\Pr_G[H\text{ is not }(\epsilon n',\gamma)\text{ edge expander}]
    \leq\sum_{s=1}^{\epsilon n'}{\left(c_1\left(\frac{s}{n'}\right)^{c_2}\right)^s}=o(1)$.
    
    More precisely, $c_1\epsilon^{c_2}<1/10$, or
    $\epsilon=(c_1c_3)^{-1/c_2}$ for some $c_3>10$.
\end{proof}

\begin{proof}[Proof of~\autoref{thm:powerlaw-permutation-vertex-expansion}]
    \label{prf:powerlaw-permutation-vertex-expansion}
    The approach is identical to~\autoref{thm:reg-vertex-expansion},
    but $d$ perfect matchings on $n$~vertices
    are replaced by a single perfect matching on $\vol(H)$ mini-vertices,
    which correspond to $n'$ vertices of $H$.
    
    When $\beta=0$, $G$ is $pn$-regular graph, so the original proof
    with~\autoref{pro:bound-prob-small-sets}
    or~\autoref{pro:bound-prob-large-sets} suffice.
    
    Otherwise, $\beta>0$.
    For any $S\subset V_H$ of size $|S|=s\leq\epsilon n'$
    and any $T\subset V_H$ of size $(1+\gamma)s$ we have:
    \begin{gather}
        d=\frac{\vol(H)}{n'}\\
        \vol(S)=sd\\
        \vol(T)=(1+\gamma)sd
    \end{gather}
    
    All mini-vertices from $S$ are matched to mini-vertices in $T$ with probability:
    
    $\Pr_{G,S,T}[N_H(S)\subseteq T]\leq\left(\frac{\vol(T)}{\vol(H)}\right)^{\vol(S)/2}$
    
    As for the number of sets $S$ of a given size (the same goes for $T$),
    $\binom{n'}{s}$ would only work for regular graphs,
    while $\binom{\vol(H)}{\vol(S)}$ would overcount because mini-vertices
    corresponding to the same vertex in $H$ are indistinguishable.
    Instead we use the trick from~\cite{gms03} by taking
    the minimum degree $d_0$ into account:
    
    $\Pr_G[\exists\text{ non-expading }S\text{ of size }s]
    \leq\binom{\vol(H)/d_0}{\vol(S)/d_0}
    \binom{\vol(H)/d_0}{\vol(T)/d_0}
    \left(\frac{\vol(T)}{\vol(H)}\right)^{\vol(S)/2}\leq$
    
    $\qquad\leq\left(\frac{en'}{s}\right)^{sd/d_0}
    \left(\frac{en'}{(1+\gamma)s}\right)^{(1+\gamma)sd/d_0}
    \left(\frac{(1+\gamma)s}{n'}\right)^{sd/2}=$
    
    $\qquad=\left(
        \left(\frac{en'}{s}\right)^{(2+\gamma)d/d_0}
        \left(\frac{1}{1+\gamma}\right)^{(1+\gamma)d/d_0}
        \left(\frac{(1+\gamma)s}{n'}\right)^{d/2}
    \right)^s=$
    
    $\qquad=\left(
        e^{(2+\gamma)d/d_0}
        (1+\gamma)^{d(1/2-(1+\gamma)/d_0)}
        \left(\frac{s}{n'}\right)^{d(1/2-(2+\gamma)/d_0)}
    \right)^s$.
    
    In order to proceed we need this exponent to be positive:
    \begin{gather}
        d(1/2-(2+\gamma)/d_0)>0\\
        0\leq\gamma<d_0/2-2
    \end{gather}
    Note that this requires $d_0>4$, while similar theorems
    \ref{thm:reg-vertex-expansion} and~\ref{thm:gms} ask for $d_0\geq 3$.
    
    When $\beta>0$, it translates into:
    \begin{gather}
        d_0=2^\beta pn^{1-\beta}>4\\
        n^{1-\beta}>2^{2-\beta}\\
        \beta<1
    \end{gather}
    
    $\Pr_G[G\text{ is not }(\alpha n,\gamma)\text{ vertex expander}]
    \leq\sum_{s=1}^{\epsilon n'}{\Pr_G[\exists\text{ non-expading }S\text{ of size }s]}=o(1)$.
    
    This final step may be achieved in two ways.
    First, we may apply~\autoref{pro:bound-prob-small-sets},
    then $\epsilon$ must be some small constant.
    Secondly, it also follows from~\autoref{pro:bound-prob-large-sets}, but we
    need to verify $c_1=o\left(n'^{c_2-1}\right)$ and $c_2>\max\{1,\log c_1\}$:
    \begin{gather*}
        c_1=e^{(2+\gamma)d/d_0}(1+\gamma)^{d(1/2-(1+\gamma)/d_0)}\\
        \log c_1=(\log e)(2+\gamma)d/d_0+(\log(1+\gamma))(d/2-(1+\gamma)d/d_0)\\
        c_2=d/2-(2+\gamma)d/d_0\\
        c_2-\log c_1=
        d/2-(2+\gamma)d/d_0
        -(\log e)(2+\gamma)d/d_0
        -(\log(1+\gamma))(d/2-(1+\gamma)d/d_0)\\
        c_2-\log c_1>0,\text{ for any }\gamma<1,
    \end{gather*}
    In this case we get $\epsilon=1/2$, but both conditions
    are satisfied only when $\gamma<1$.
\end{proof}

\subsection{Error from Rounding Degrees}
\label{subsec:powerlaw-permutation-rounding-error}

To get the exact degree sequence, one needs to round down
the expression~\eqref{eq:powerlaw-permutation-deg} for $\deg(v)$.
In this case the error of $\vol(G)$ is at most $n$.
For our choice of $d_0$ this error is negligible compared to $\vol(H)$
when $0\leq\beta\leq1$, but it might be significant otherwise:
\begin{equation}
    \vol(H)=
    \begin{cases}
        \Theta\left(n^2\right) & \quad \text{if } \beta=0,\\
        \Theta\left(n^{2-\beta}\right) & \quad \text{if } 0<\beta<1,\\
        \Theta(n \log n) & \quad \text{if } \beta=1,\\
        \Theta(n) & \quad \text{if } \beta>1.
    \end{cases}
\end{equation}

\section{\texorpdfstring{``Octopus''}{"Octopus"} Graphs}

The core of the ``octopus'' graph is defined as an induced subgraph
with all the vertices of degree at least $d_0=n^{1/\log\log n}$.

%The size of the core is $n^{c/\log\log n}$ and the diameter is $O(\log\log n)$.

%Almost all vertices are within distance $O(\log\log n)$ of the core
%although there are vertices at distance $\Theta(\log n)$ from the core. (explain that it's for diam)

The number of vertices of degree more than $d$
is about $\left(\frac{\beta-2}{\beta-1}\right)^{\beta-1}n$.
\begin{gather*}
    w_x=\frac{\beta-2}{\beta-1}dn^{\frac{1}{\beta-1}}x^{-1/(\beta-1)}>d\\
    \frac{\beta-2}{\beta-1}n^{\frac{1}{\beta-1}}>x^{1/(\beta-1)}\\
    x<\left(\frac{\beta-2}{\beta-1}\right)^{\beta-1}n
\end{gather*}

Analogously, $i_1$ is the index of the last vertex included in the core:
\begin{gather*}
    w_x=\frac{\beta-2}{\beta-1}dn^{\frac{1}{\beta-1}}x^{-1/(\beta-1)}\geq n^{1/\log\log n}=d_0\\
    \frac{\beta-2}{\beta-1}dn^{\frac{1}{\beta-1}-1/\log\log n}\geq x^{1/(\beta-1)}\\
    x\leq\left(\frac{\beta-2}{\beta-1}d\right)^{\beta-1}n^{1-\frac{\beta-1}{\log\log n}}=i_1
\end{gather*}

$i_1\leq n$ implies a restriction on the average degree:
\begin{equation}
    d\leq\frac{\beta-1}{\beta-2}n^{1/\log\log n}
\end{equation}

By our assumptions: $i_1=i_0\left(\frac{d_{max}}{n^{1/\log\log n}}\right)^{\beta-1}\gg i_0$.

Let $n'$ denote the size of the core.

%\cite{cl04} claims that:
%\begin{gather}
%    n'=n^{c/log\log n}\\
%    \log n'=c\log n/\log\log n
%\end{gather}
%Alternatively, we can say:
\begin{gather}
    n'=i_1-i_0\\
    \log n'\approx\log i_1\approx\log n\left(1-\frac{\beta-1}{\log\log n}\right)
\end{gather}

The average degree of the core:

$\frac{\sum_{k=i_0}^{i_1}{w_i}}{i_1-i_0}
\approx\frac{\beta-2}{\beta-1}dn^{\frac{1}{\beta-1}}\frac{1}{i_1}
\sum_{k=1}^{i_1}{k^{-1/(\beta-1)}}
\approx\frac{\beta-2}{\beta-1}dn^{\frac{1}{\beta-1}}\frac{1}{i_1}
\frac{i_1^{1-1/(\beta-1)}}{(\beta-2)/(\beta-1)}=$

$\qquad=d\left(\frac{n}{i_1}\right)^{\frac{1}{\beta-1}}
=d\left(\left(\frac{\beta-2}{\beta-1}d\right)^{\beta-1}n^{-\frac{\beta-1}{\log\log n}}\right)^{-\frac{1}{\beta-1}}
=d\left(\frac{\beta-2}{\beta-1}d\right)^{-1}n^{1/\log\log n}=$

$\qquad=\frac{\beta-1}{\beta-2}n^{1/\log\log n}=\omega(\log n')$

As a result, the core is edge expander w.h.p. by~\autoref{lem:coin-toss-lemma-d}.

Moreover, this core contains $G(n'',p)$ by~\cite{cl04}:

$n''p=\left(cnd_0^{1-\beta}\right)\frac{d_0^2}{dn}
=\left(cn^{1+(1-\beta)/\log\log n}\right)\frac{n^{2/\log\log n}}{dn}
=\frac{c}{d}n^{(3-\beta)/\log\log n}=$

$\qquad=\frac{\beta-2}{\beta-1}n^{\frac{1}{\beta-1}+(3-\beta)/\log\log n}
=\omega(\log n'')$
\begin{equation}
    \log n''=\log d+\log n(\frac{1}{\beta-1}+1+(1-\beta)/\log\log n)
\end{equation}
%$\frac{\log n''}{\log pn''}=\frac{\log n}{\log n/\log\log n}=\log\log ns$

\cite{cl04} uses this fact to prove $O(\log\log n)$ diameter of the core.

Thus, we can apply~\autoref{thm:kri-gnp} to get large vertex expander inside of the core,
which could serve as an alternative step for the proof of $O(\log n)$ diameter
of the ``octopus'' graph.

\section{Diameter of Vertex Expanders}

In this section we focus on $(\epsilon n,\gamma)$ vertex expanders,
especially on the case $\epsilon<1/2$.

First, consider an example: a graph $G$ that consists of two disconnected
$(n/4,\gamma)$ vertex expanders of size $n/2$ each.
Any subset of $G$ of size up to $n/4$ is the union of subsets of these two parts,
and it expands proportionally.
Therefore, $G$ is $(n/4,\gamma)$ vertex expander as well,
and yet it is not connected.

In general, when $\epsilon\leq1/4$, the connectivity of
$(\epsilon n,\gamma)$ vertex expander is not guaranteed.

A well known fact about $(n/2,\gamma)$ vertex expanders is that
they have diameter $O(\log n)$, see for example~\cite{rao12,hlw06}.
We extend on this by demonstrating that the diameter is still $O(\log n)$
even for small constant $\epsilon<1/2$, assuming the graph is connected.

\begin{theorem}
    \label{thm:vertex-expander-diameter}
    Let $G=(V,E)$ be a connected $(s(n),\gamma)$ vertex expander of size $n$,
    where $s(n)\leq n/2$ and $\gamma=\Omega(1)$.
    Then the diameter of $G$ is $O\left(\frac{n}{s(n)}\log s(n)\right)$.
\end{theorem}

\begin{corollary}
    The diameter of a connected $(\epsilon n,\gamma)$ vertex expander of size $n$,
    where $\epsilon\leq1/2$ and $\gamma$ are some positive constants, is $O(\log n)$.
\end{corollary}

Similarly, $s(n)=\Theta\left(\frac{n}{\log n}\right)$ gives us
diameter $O\left(\log^2n\right)$,
and $s(n)=\Theta(\sqrt{n})$ gives $O\left(\sqrt{n}\log n\right)$.
We now prove an intermediate result needed for the theorem.

\begin{lemma}
    \label{lem:vertex-expander-partitioning}
    If $G=(V,E)$ is a connected $(s(n),\gamma)$ vertex expander,
    then there exists a partitioning $\{S_1,\ldots,S_k\}$ of $V$,
    such that $k=O(n/s(n))$ and each $S_i$ has diameter $O(\log s(n))$.
\end{lemma}

\begin{proof}
    Let $B(v,r)=\{u\in V\;|\;\dist(v,u)\leq r\}$ denote
    a ball of radius $r\geq 0$ around vertex $v$.
    
    For any $v\in V$ and $r\geq 1$, if $|B(v,r-1)|\leq s(n)$,
    then $|B(v,r)|\geq\min\left\{s(n),(1+\gamma)^r\right\}$
    by the expansion property of $G$.
    As all subsets of size up to $s(n)$ are expanding, there exists
    \begin{equation}
        r_0=\ceil{\log_{1+\gamma}s(n)}=\Theta(\log s(n))
    \end{equation}
    such that $|B(v,r_0)|\geq s(n)$. The diameter of $B(v,r_0)$ is $O(r_0)$.
    
    We maintain an invariant that all $S_i$ are disjoint
    and have diameter $O(r_0)$ throughout the following partitioning process.
    
    Begin with arbitrary $v_1\in V$ and select $S_1=B(v_1,r_0)$.
    As long as there is some other vertex $v$ whose $B(v,r_0)$ contains
    at least $s(n)/2$ of not yet selected vertices,
    add these new vertices to the next set $S_i$.
    We will stop after $k\leq2n/s(n)$ steps.
    Every remaining vertex $u$ must be within distance $r_0$ from some $S_j$,
    so we add each such $u$ to the corresponding~$S_j$.
\end{proof}

\begin{proof}[Proof of~\autoref{thm:vertex-expander-diameter}]
    \autoref{lem:vertex-expander-partitioning} gives us
    the partitioning $\{S_1,\ldots,S_k\}$ of $V$.
    We create a~graph $G'$ of size $k$ from $G$ by contracting each $S_i$
    into a single vertex and merging multiple edges.
    Note that $G'$ is connected by construction.
    
    Now consider arbitrary $u,v$ from $G$. Clearly, the distance between
    their corresponding vertices in $G'$ is at most $k$.
    Let $D$ be the maximum diameter of any $S_i$.
    Then $\dist(u,v)$ in $G$ is at most $k(2D+1)$
    which is exactly $O\left(\frac{n}{s(n)}\log s(n)\right)$.
\end{proof}
