\chapter{Random Power-Law Graphs}
\label{ch:powerlaw}

\section{Models of Graphs}

We will work with models of graphs that define either degree sequences or frequencies of degrees.
In the latter case one can easily obtain the degree sequence as well,
e.g., the frequencies $(x_1,x_2,x_3)=(2,1,4)$ would be transformed into the degrees $(1,1,2,3,3,3,3)$.

Given a degree sequence $(w_1,\ldots,w_n)$, there are several ways to generate corresponding random graphs $G=(V,E)$~\cite{hop08}.
Depending on the particular model, this will be either exact (fixed) or expected degree sequence.

All subsequent models possibly result in pseudographs --- multigraphs with self-loops.
Therefore, the degree sequences are only required to be pseudographic:
degrees must be non-negative, and in case of exact degrees, they should add up to an even number~\cite{hak62}.

\subsection{Coin Toss Model}

For each pair of vertices $i$ and $j$, we include the edge $(i,j)$ in the graph with probability
\begin{equation}
    p_{ij}=\Pr_G[(i,j)\in E]=
    \begin{cases}
        \frac{w_i w_j}{\sum_{k\in V}{w_k}} & \quad \text{if }i\neq j,\\
        \frac{w_i^2}{2\sum_{k\in V}{w_k}} & \quad \text{otherwise}.
    \end{cases}
\end{equation}
Given expression is a valid probability, if we have
\begin{equation}
    \label{eq:coin-toss-assumption}
    \max_k w_k^2<\sum_{k\in V}{w_k}
\end{equation}
% http://mathworld.wolfram.com/GraphicSequence.html
This assumption also implies that the sequence is graphic,
i.e., realizable by some graph.

The standard convention is that self-loop edges contribute 2 to the degrees of the nodes.
Considering this, we show that the model is well-defined:
\begin{equation}
    \E_G[\deg(i)]=\sum_{j\neq i}{p_{ij}}+2p_{ii}
    =\frac{\sum_{j\neq i}{w_i w_j}+w_i^2}{\sum_{j\in V}{w_j}}=w_i
\end{equation}

Alternatively, one can generate a graph by including some number of edges between $i$ and $j$
according to a Poisson process with mean $\lambda=\frac{w_i w_j}{\sum_{k\in V}{w_k}}$.
High degree vertices would then form a clique.

In this model all edges are independent, and $(w_1,\ldots,w_n)$ are the expected degrees.

\subsection{\texorpdfstring{$m$}{m} Edges Model}

Uniformly at random we generate $m=\frac{1}{2}\sum_{k\in V}{w_k}$ edges by
selecting vertices for endpoints with probability proportional to their degrees.
In this case $w_i$'s are also expected degrees, however, we get
the exact number of edges, which are no longer independent.

\subsection{Permutation Model}
\label{subsec:permutation-model}

We create a sequence with $w_i$ mini-vertices for each vertex $i$,
randomly permute it, and connect each consecutive
non-intersecting pair of mini-vertices with an edge.
Equivalently, one can choose a uniformly random perfect matching on this sequence.
Then we put an edge between vertices $i$ and $j$ if at least one pair
of mini-vertices corresponding to both of them is adjacent.

This model guarantees the exact number of edges and they are not independent,
but the degree sequence is now exact.

\subsection{Full Graph Selection Model}

We randomly pick a graph $G$ from a family of all graphs with given
exact degree sequence, but it is usually not practical.

In this chapter we will examine the properties of power-law graphs under different models.

\section{Coin Toss Power-Law Model}
\label{sec:powerlaw-coin-toss-model}

We are given the expected degree sequence $(w_1,\ldots,w_n)$, where $w_v=\E[\deg(v)]$.
\begin{gather}
    \label{eq:powerlaw-coin-toss-edge-pr}
    \Pr_G[(u,v)\in E]=\frac{w_u w_v}{\Vol(G)},
    \quad\text{ assuming }\max_v w_v^2<\Vol(G)\\
    \Vol(G)=\E_G[\vol(G)]=\E_G\left[\sum_{v\in V} \deg(v)\right]=\sum_{v\in V} w_v
\end{gather}

%Note that by~\eqref{eq:chernoff-combined}: $\Pr_G[|\vol(G)-\Vol(G)|\geq\delta]\leq2\,\exp(-\delta^2/(3\;\Vol(G)))$.

Let the expected number of vertices with degree $x$ be equal $y$,
which follows the power-law distribution for some super-constant $\alpha>0$ and constant $\beta>0$:
\begin{equation}
    \E_G[|\{v \in V\;|\;\deg(v)=x\}|]=y=\frac{e^\alpha}{x^\beta},
    \quad\text{where }x\in\{1,\ldots,d_{max}\}.
\end{equation}

As mentioned earlier, this model is asymptotically identical
to the one from~\cite{acl01}, whose properties are summarized in~\autoref{tab:graphs-properties-acl01}.

\begin{table}
    %\begin{shaded}
    \begin{center}
        \renewcommand*{\arraystretch}{1.3}
        \begin{tabular}{|l|c|c|c|c|c|c|}
            \hline
            \multicolumn{1}{|r|}{$\beta$} & $(0;1)$ & $1$ & $(1;2)$ & $2$ & $(2;\beta_0)$ & $(\beta_0;\infty)$\\
            \hline\hline
            {\renewcommand{\arraystretch}{1.0}\begin{tabular}{@{}l@{}}the largest\\component\end{tabular}} & $n$ & \multicolumn{4}{|c|}{the giant component, $\Theta(n)$} & \multirow{2}{*}{\vspace{-4mm}$O\left(n^{2/\beta}\log n\right)$}\rule{0pt}{19pt}\\[0.7em]
            \cline{1-6}
            {\renewcommand{\arraystretch}{1.0}\begin{tabular}{@{}l@{}}the 2nd largest\\component\end{tabular}} & --- & \multicolumn{2}{|c|}{$O(1)$} & $O\left(\frac{\log n}{\log\log n}\right)$ & $O(\log n)$ &\rule{0pt}{20pt}\\[0.7em]
            \cline{1-7}
            $|V|=n$ & $\frac{e^{\alpha/\beta}}{1-\beta}$ & $\alpha e^\alpha$ & \multicolumn{4}{|c|}{$\zeta(\beta)e^\alpha$}\rule{0pt}{21pt}\\[0.7em]
            \cline{1-7}
            $|E|$ & \multicolumn{3}{|c|}{$\frac{1}{2}\frac{e^{2\alpha/\beta}}{2-\beta}$} & $\frac{1}{4}\alpha e^\alpha$ & \multicolumn{2}{|c|}{$\frac{1}{2}\zeta(\beta-1)e^\alpha$}\rule{0pt}{21pt}\\[0.7em]
            \hline
        \end{tabular}
        \caption[Properties of power-law graphs from~\cite{acl01}.]
                {Properties of power-law graphs from~\cite{acl01}, $\beta_0\approx3.4785$.}
        \label{tab:graphs-properties-acl01}
    \end{center}
    %\end{shaded}
\end{table}

\subsection{Maximum Degree, Expected Size, and the Number of Edges}

Like~\cite{acl01}, we ignore isolated vertices, $y\geq1$
and $0\leq\ln y=\alpha-\beta\ln x$, consequently:
\begin{equation}
    d_{max}=e^{\alpha/\beta}
\end{equation}

The size of the graph and the expected number of edges:
\begin{equation}
    \label{eq:powerlaw-coin-toss-n-e}
    n=\sum_{x=1}^{e^{\alpha/\beta}}{\frac{e^\alpha}{x^\beta}}
    \qquad \E_G[|E|]=\frac{\Vol(G)}{2}=\frac{1}{2}\sum_{x=1}^{e^{\alpha/\beta}}{x\frac{e^\alpha}{x^\beta}}
\end{equation}

%These values of $d_{max}$ and $\Vol(G)$ do not satisfy~\eqref{eq:coin-toss-assumption}:
%
%$d_{max}^2=e^{2\alpha/\beta}>\Vol(G)$, when $0<\beta<1$ or $\beta=2$
%
%Erd\H{o}s-Gallai theorem (for simple graphs with integer $w_i$'s) gives the necessary and sufficient condition:
%$\sum_{i=1}^{k}{w_i}\leq k(k-1)+\sum_{i=k+1}^{n}{\min(w_i,k)}$.
%
%The number of edge-vertex adjacencies among $k$ vertices of the highest degree$\leq\ldots$
%
%$n-1\geq w_1\geq w_2\geq\ldots\geq w_n$,
%$\sum_{i=1}^n{w_i}$ is even

To get some intuition about the density of such graphs,
we consider the special case when $d_{max}=n$:
\begin{gather}
    n=\sum_{x=1}^{n}\frac{e^\alpha}{x^\beta}=e^\alpha H_{n,\beta} \qquad e^\alpha=\frac{n}{H_{n,\beta}}\\
    \E_G[m]=\frac{1}{2}\E_G[\sum_{v \in V}\deg(v)]=\frac{1}{2}\sum_{x=1}^{n}x\frac{e^\alpha}{x^\beta}
    =\frac{e^\alpha}{2}\sum_{x=1}^{n}x^{1-\beta}=\frac{e^\alpha}{2}H_{n,\beta-1}
    =\frac{n}{2}\frac{H_{n,\beta-1}}{H_{n,\beta}}
\end{gather}

$H_{n,-1}=\frac{n(n+1)}{2},
\qquad H_{n,0}=n,
\qquad H_{n,1}=\log n+O(1),
\qquad H_{n,\beta}=O(1)\text{, if }\beta>1.$

\begin{equation}
    \E_G[m]=
    \begin{cases}
        \Theta\left(n^2\right) & \quad \text{if } \beta=0,\\
        \Theta\left(n^2/\log n\right) & \quad \text{if } \beta=1,\\
        \Theta(n\log n) & \quad \text{if } \beta=2,\\
        \Theta(n) & \quad \text{if } \beta>2.
    \end{cases}
\end{equation}

\subsection{Expected Volume and Average Degree}

When $n\to\infty$, the expected volume $\Vol(G)$ is well-defined only if $\beta>2$:
\begin{gather}
    \Vol(G)\approx\int_1^{\infty}{x\frac{e^\alpha}{x^\beta}}\,dx
    =\frac{e^\alpha}{2-\beta}\left[ x^{-\beta+2} \right]_1^\infty
    =\frac{e^\alpha}{\beta-2}
\end{gather}
On the other hand, considering $d_{max}=e^{\alpha/\beta}$ we get:
\begin{equation}
    \label{eq:powerlaw-coin-toss-vol}
    \Vol(G)\approx\frac{e^\alpha}{2-\beta}\left(\left(e^{\alpha/\beta}\right)^{-\beta+2}-1\right)
    =\frac{1}{2-\beta}\left(e^{2\alpha/\beta}-e^\alpha\right)
\end{equation}
The expected average degree:
\begin{equation}
    d=\frac{\Vol(G)}{n}
\end{equation}
The second-order average degree:
\begin{equation}
    \tilde{d}=\frac{\sum_{v\in V}{w_v^2}}{\Vol(G)}\approx
    \begin{cases}
        d\frac{(\beta-2)^2}{(\beta-1)(\beta-3)} & \quad \text{if } \beta>3\\
        \frac{1}{2}d\ln \frac{2d_{max}}{d} & \quad \text{if } \beta=3\\
        d^{\beta-2}\frac{(\beta-2)^{\beta-1}d_{max}^{3-\beta}}{(\beta-1)^{\beta-2}(3-\beta)} & \quad \text{if } 2<\beta<3
    \end{cases}
\end{equation}

%\subsection{The Least and the Most Popular Vertices}
%
%$L_c = \{c\text{ least popular vertices}\}, M_c = \{c\text{ most popular vertices}\}$
%OR "vertices w/ deg<=c"?
%
%$|L_c|=\sum_{x=1}^{c}\E[|\{v\in V\;|\;\deg(v)=x\}|]
%\qquad |M_c|=...$
%
%\textbf{\#surviving edges, connectivity} $|E(L_c)|=...$
%
%$|E(L_c)|=$ like $\E[m]$ w/ limited sum \& prob. that edge stays in subset (via $\Vol(S)$)
%
%Connected components: 2nd largest<M (power-law),
%or spanning tree \& matrix-tree thm (general expected degree sequences)
%
%$L_c$ is probably not a good expander.
%Look for expander in $M_c$.
%:TODO:
%
%\subsection{Dependence of Properties on \texorpdfstring{$\beta$}{b}}
%
%:TODO: \textbf{How do properties depend on beta? 0$\rightarrow$uniform}

\section{Permutation Power-Law Model}

In the permutation model each vertex $v\in\{1,\ldots,n\}$ has exactly
$w_v$ neighbors chosen with probability proportional to their degrees:
\begin{equation}
    \label{eq:powerlaw-permutation-deg}
    \deg(v)=w_v=\frac{pn}{v^\beta},\quad 0<p\leq 1
\end{equation}

If $\beta=0$, the coin toss model with expected degree sequence
$(w_1,\ldots,w_n)$ would be identical to $G(n,p)$ with self-loops:
\begin{gather*}
    w_v=pn
    \qquad \vol(G)=pn^2
    \qquad|E|=\frac{p n^2}{2}\\
    \Pr_G[(u,v)\in E]=\frac{w_u w_v}{\sum_i w_i}=\frac{(pn)^2}{pn^2}=p
\end{gather*}

Note that the degrees $w_i$ should be integers so we have
to round down~\eqref{eq:powerlaw-permutation-deg}. The issue of emerging error
is addressed in~\autoref{subsec:powerlaw-permutation-rounding-error}.

\section{k-CNF Power-Law Model}
\label{sec:powerlaw-k-cnf-model}

Generation of CNF $f$ having $m$ clauses with $k$ literals each is done by adding variable $i\in\{1,\ldots,n\}$
with a random sign to the current clause $C\in\{C_1,\ldots,C_m\}$ at a position $t$, $1\leq t\leq k$, with probability
$p_i=\frac{w_i}{\sum_{j=1}^{n}{w_j}}$, where $w_i=i^{-\beta}$~\cite{abl09}.

If any resulting clause is a tautology (it contains both some variable and its negation) or simplifiable (there is the same literal at multiple positions),
then it is discarded and regenerated from scratch. Essentially it means that we are choosing variables for each clause without replacement.

\textit{The variable incidence graph (VIG)} $G(f)=(V,E)$ is defined on $V=\{1,\ldots,n\}$,
and edge $(i,j)$ is added to $E$ whenever both $i$ and $j$ appear in the same clause.
The signs of the literals are ignored in VIG, so we will ignore them as well,
writing ``$i\in C$'' instead of ``$i\in C\lor-i\in C$''.
\begin{gather}
    \Pr_{G(f)}[(i,j)\in E]=1-\prod_C{(1-\Pr[i\in C\land j\in C])},\text{ for }i\neq j
\end{gather}

Let's assume $k=3$ for simplicity.

$\Pr[i\in C\land j\in C]=\sum_{\substack{1\leq x\leq n\\x\notin\{i,j\}}}\left(
    \Pr[C=(i,j,x)]+\Pr[C=(j,i,x)]+\ldots+\Pr[C=(x,j,i)]
\right)$

$p_x\leq1-p_i-p_j$

$\Pr[C=(i,j,x)]=p_i\frac{p_j}{1-p_i}\frac{p_x}{1-p_i-p_j}$

$\Pr[C=(i,x,j)]=p_i\frac{p_x}{1-p_i}\frac{p_j}{1-p_i-p_x}$, and so on.

\subsection{Noncentral Hypergeometric Distribution}
Sampling of elements from different classes with weights without replacement
is captured by \textit{Wallenius' noncentral hypergeometric distribution}.

We have $\sum_{i=1}^c{m_i}$ objects total from $c$ classes,
$m_i$ is the number of objects of class $i$.
$n$ objects are sampled without replacement.
The probability that an object from class $i$ is sampled is proportional to $w_i$.

If at step $t$ we have sampled $(x_{1,t},\ldots,x_{c,t})$ objects,
i.e., exactly $x_{i,t}$ objects from class $i$,
then the probability that the next draw gives an object from class $i$ is:
\begin{equation}
    \frac{(m_i-x_{i,t})w_i}{\sum_{j=1}^c{(m_j-x_{j,t})w_j}}
\end{equation}

For the multivariate case:
\begin{gather}
    \text{mwnchypg}(\mathbf{x};n,\mathbf{m},\mathbf{w})=
    \left(\prod_{i=1}^c{\binom{m_i}{x_i}}\right)
    \left(\int_0^1{\prod_{i=1}^c{(1-t^{w_i/d})^{x_i}}\,dt}\right)\\
    d=\mathbf{w}(\mathbf{m}-\mathbf{x})=\sum_{i=1}^c{w_i(m_i-x_i)}
\end{gather}
The probability function can be calculated using a variety of methods:
recursive calculation, binomial expansion methods,
Taylor expansion methods, continued fraction expansion,
and numerical integration~\cite{fog08}.
At the same time direct formal analysis does not look promising.
Instead, we can use the fact that the sequence of expected degrees
in VIG $G(f)$ also follows power-law distribution.

\begin{theorem}[Theorem 1 from~\cite{abl09}]
    In the power-law model with continuous probability distribution
    $\phi(x,\beta)=(1-\beta)x^{-\beta}$, when $n\to\infty$:
    \begin{equation}
        \Pr_f[\text{variable }i\text{ has }k\text{ occurrences}]\propto k^{-\alpha},
        \quad\text{where }\alpha=1/\beta+1.
    \end{equation}
\end{theorem}

Let $x_i$ be the number of occurrences/clauses with variable $i$.
\begin{equation}
    \Pr[x_i=k]=ck^{-\alpha}
\end{equation}

Events ``occurrence of $i$ in clause $C$'' are independent for all clauses,
so the actual number of occurrences will be close to its expectation:

$\E[x_i]=\sum_{k=1}^m{k\Pr[x_i=k]}=\sum_{k=1}^m{ck^{1-\alpha}}=cH_{m,\alpha-1}$

For any pair $(i,j)$, if $x_i+x_j\geq m$, then $\Pr[(i,j)\in E]=1$ by the pigeonhole principle.
Otherwise $x_i+x_j<m$ and:

$\Pr_{G(f)}[(i,j)\in E]=\Pr[\exists C:i\in C\land j\in C]
=1-\frac{\binom{m-x_i}{x_j}}{\binom{m}{x_j}}=$

$\qquad=1-\frac{(m-x_i)!}{(m-x_i-x_j)!}\frac{(m-x_j)!}{m!}=$

$\qquad=1-\prod_{k=0}^{x_j-1}{\frac{m-x_i-k}{m-k}}$.

\section{\texorpdfstring{``Octopus''}{"Octopus"} Power-Law Graph Model}
\label{sec:powerlaw-octopus-model}

This expected degree model is found in~\cite{cl04}:
\begin{gather}
    \E_G[\deg(i)]=w_i=ci^{-1/(\beta-1)}\\
    i_0\leq i<n+i_0\\
    c=\frac{\beta-2}{\beta-1}dn^{\frac{1}{\beta-1}}\\
    i_0=n\left(\frac{d(\beta-2)}{d_{max}(\beta-1)}\right)^{\beta-1}
\end{gather}
The known results about the average distance and the diameter of resulting graphs
are collected in~\autoref{tab:graphs-properties-cl04}.

\begin{table}
    \begin{center}
        \renewcommand*{\arraystretch}{1.3}
        \begin{tabular}{|l|c|c|c|}
            \hline
            \multicolumn{1}{|r|}{$\beta$} & $(2;3)$ & $3$ & $(3;\infty)$\\
            \hline
            \multicolumn{1}{|r|}{$\beta'$} & $(1/2;1)$ & $1/2$ & $(0;1/2)$\\
            \hline\hline
            average distance & $O(\log\log n)$ & $\Theta(\log n/\log\log n)$ & $\Theta(\log n/\log \tilde{d})$\\
            \hline
            diameter & \multicolumn{3}{|c|}{$\Theta(\log n)$}\\
            \hline
        \end{tabular}
        \caption[Properties of power-law graphs from~\cite{cl04}.]
                {Properties of power-law graphs from~\cite{cl04},\\$\beta'=\frac{1}{\beta-1}$ is the actual exponent.}
        \label{tab:graphs-properties-cl04}
    \end{center}
\end{table}

We consider the same specific family of such graphs, as in~\cite{cl04}, i.e.:
\begin{gather}
    2<\beta<3\implies\frac{1}{2}\leq\frac{1}{\beta-1}\leq 1\\
    d>1\\
    d_{max}\gg d\\
    \log d_{max}\gg \log n/\log\log n
\end{gather}

The graphs are generated like in the coin toss model
and \eqref{eq:coin-toss-assumption} is assumed to be true.

Note that $d_{max}$ and $d$ are simply parameters of this model.

The maximum degree $w_{i_0}=\frac{\beta-2}{\beta-1}dn^{\frac{1}{\beta-1}}i_0^{-1/(\beta-1)}=d_{max}$.

And the average degree is $\frac{1}{n}\sum_{k=i_0}^{n+i_0}{w_i}
\approx\frac{1}{n}\sum_{k=1}^{n}{w_i}
=\frac{1}{n}\sum_{k=1}^{n}{
    \frac{\beta-2}{\beta-1}dn^{\frac{1}{\beta-1}}k^{-1/(\beta-1)}
}=$

$\qquad=\frac{\beta-2}{\beta-1}\frac{d}{n^{(\beta-2)/(\beta-1)}}
\sum_{k=1}^{n}{k^{-1/(\beta-1)}}\approx$
    
$\qquad\approx\frac{\beta-2}{\beta-1}\frac{d}{n^{(\beta-2)/(\beta-1)}}
\frac{n^{1-1/(\beta-1)}}{1-1/(\beta-1)}=d$.

\subsection{The Second-Order Average Degree}

\cite{cl04} only states the final result, so we provide
the derivation here for completeness.

$\tilde{d}=\frac{1}{dn}\sum_{k=1}^{n}{w_i^2}
=\frac{1}{dn}\sum_{k=1}^{n}{
    \left(\frac{\beta-2}{\beta-1}\right)^2d^2n^{\frac{2}{\beta-1}}k^{-2/(\beta-1)}
}=$

$\quad=\left(\frac{\beta-2}{\beta-1}\right)^2\frac{d}{n^{1-2/(\beta-1)}}
\sum_{k=1}^{n}{k^{-2/(\beta-1)}}$

\textbf{Case $\beta>3$:}

$\tilde{d}\approx\left(\frac{\beta-2}{\beta-1}\right)^2\frac{d}{n^{1-2/(\beta-1)}}
\frac{n^{(\beta-3)/(\beta-1)}}{(\beta-3)/(\beta-1)}
=\frac{(\beta-2)^2}{(\beta-1)(\beta-3)}d$

\textbf{Case $\beta=3$:}

$i_0=n\left(\frac{d}{2d_{max}}\right)^2$

$\tilde{d}\approx\frac{d}{4}\sum_{k=i_0}^{n}{k^{-1}}\approx\frac{d}{4}(\ln n-\ln i_0)
=\frac{d}{4}(\ln n-\ln n-2\ln\frac{d}{2d_{max}})=\frac{d}{2}\ln\frac{2d_{max}}{d}$

\textbf{Case $2<\beta<3$:}

$\tilde{d}\approx\left(\frac{\beta-2}{\beta-1}\right)^2\frac{d}{n^{(\beta-3)/(\beta-1)}}
\sum_{k=i_0}^{n}{k^{-2/(\beta-1)}}\approx$

$\qquad\approx\left(\frac{\beta-2}{\beta-1}\right)^2\frac{d}{n^{(\beta-3)/(\beta-1)}}
\left(\frac{n^{(\beta-3)/(\beta-1)}}{(\beta-3)/(\beta-1)}
    -\frac{i_0^{(\beta-3)/(\beta-1)}}{(\beta-3)/(\beta-1)}
\right)\approx$

$\qquad\approx-\frac{(\beta-2)^2}{(\beta-1)(\beta-3)}d
\left(\frac{d(\beta-2)}{d_{max}(\beta-1)}\right)^{\beta-3}
=\frac{(\beta-2)^{\beta-1}}{(\beta-1)^{\beta-2}(3-\beta)}
d^{\beta-2}d_{max}^{3-\beta}$.

\subsection{Diameter}

\begin{theorem}[Theorem 2.6 from~\cite{cl04}]
    Suppose a power-law random graph with exponent $\beta$ has average degree $d>1$
    and maximum degree $d_{max}\gg n^{1/\log\log n}$.
    If $2<\beta<3$, its diameter is $\Theta(\log n)$ w.h.p.
\end{theorem}

\begin{proof}[Proof outline]
    The core of the ``octopus'' graph is defined to contain all vertices of degree
    at least $n^{1/\log\log n}$.
    Combining the following claims gives $O(\log n)$ diameter of the graph w.h.p.
    
    Claim 1: The diameter of the core is $O(\log\log n)$ w.h.p.
    
    Claim 2: Almost all vertices with degree at least $\log n$ are
    within distance $O(\log\log n)$ from the core.
    
    Claim 3: Each vertex in the giant connected component is within distance
    $O(\log n)$ from a vertex of degree at least $O(\log n)$.
\end{proof}

Such a specific structure explains the name of these graphs.

%``Local sparsity'' property from~\cite{kri17} should be called ``not being locally sparse''.
%Can we ``shave'' some extra edges off from very dense spots and apply~\autoref{thm:kri}?
%
%So $G\backslash L_c$ (popular nodes) has expander, and $M_c$ is its core (most popular nodes),
%we want everyone in $G\backslash L_c$ to be connected to the core
%(and $L_c$ might have sparse spots not connected to core),
%then no matter how we split the core in some cut, there will be enough edges crossing this cut.
%
%Edge is in the transitive closure of $G\iff$ its endpoints are connected in $G$.
%$A^2$ (includes the edges for all paths of length $\leq 2$) is not yet transitive closure (which can also use new edges).
