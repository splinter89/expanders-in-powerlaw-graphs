\chapter{Comparison of the Graph Models}
\label{ch:comparison}

We have briefly compared our findings to some previous papers in~\autoref{ch:intro}.
Now we would like to address more correspondences, both quantitative and qualitative.

\section{Overview of the Expansion Properties}

As usual, we denote the size of the subgraphs $n'$,
the minimum degree $d_0$ and the average degree $d$,
the expansion $\gamma$, and $\epsilon,\delta>0$ are some arbitrary small constants.

\subsection{Previously Known Results}

We saw in~\autoref{ch:prelims} that random $d$-regular graphs are
themselves $(\epsilon n,d-2-\delta)$ vertex and edge expanders.

The model from Gkantsidis et al.~\cite{gms03} is similar to our coin toss model
and it describes the graphs with a small constant conductance $0.0175$,
which is a generalization of $(n/2,\gamma)$ edge expansion.

Lastly, $G(n,p)$ graphs contain $(n'/2,\Theta(1))$ vertex expanders of linear size~\cite{kri17}.

\subsection{Our Results}

Graphs in the coin toss model contain $(n'/2,d/2-\delta)$ edge expanders.
The size of these subgraphs is $n'=\Theta(n)$ when $\beta\leq 1.6$,
but only $n'=\Theta\left(n^{1/\beta}\right)$ for larger $\beta$.

In the permutation model, on the other hand, we were able to prove
the existence of $(n'/2,d/2)$ edge expanders if $\beta\leq 1.72$,
and also both $(n'/2,1-\delta)$ and $(\epsilon n',d_0/2-2-\delta)$
vertex expanders for $0<\beta<1$.
The trade-off between the maximum size of expanding subsets
and the expansion rate is the most apparent in this model.
And as we know, any vertex expansion would imply the same edge expansion,
but unluckily, we couldn't secure even weak vertex expansion for $\beta>1.72$.

\section{Uniform Random Graphs}

Although power-law graphs have less uniform structure than $G(n,p)$,
we produced evidence that they are still random enough because of
the large expanding subgraphs.

$G(n,p)$ graphs have a sharp threshold for connectivity~\cite{er59} and
become connected w.h.p. when $p=(1+\epsilon)\log n/n$, for $\epsilon>0$,
that is, when the expected degrees are $d\approx pn>\log n$.
It harmonizes with our \autoref{lem:coin-toss-lemma-d}, which explains
that graphs with the expected average degree $d>10\log n$
are $(n/2,d/2-\delta)$ edge expanders, for any small constant $\delta>0$.

\autoref{thm:kri-gnp} adds to this by saying that $G(n,p)$ graph
in supercritical regime with $p=(1+\epsilon)/n$ w.h.p. contains
$(n'/2,\gamma)$ vertex expander on $n'=\Theta(n)$ vertices~\cite{kri17}.

The parameters are
$c_1=1+\epsilon,c_2=1+\epsilon^2/10$,
$\alpha=\left(\frac{c_2}{5c_1}\right)^{c_2/(c_2-1)}$,
and $\Delta=4\ln\frac{1}{\epsilon}$.

In this case it guarantees the vertex expansion
\begin{equation}
    \gamma=\frac{c_1-c_2}{\Delta\log_2{\frac{1}{\alpha}}}
    =\frac{\epsilon-\epsilon^2/10}{
        \left(4\ln\frac{1}{\epsilon}\right)
        \frac{c_2}{c_2-1}
        \log_2{\frac{5c_1}{c_2}}
    }
    =\frac{\epsilon-\epsilon^2/10}{
        \left(4\ln\frac{1}{\epsilon}\right)
        \frac{1+\epsilon^2/10}{\epsilon^2/10}
        \log_2{\frac{5c_1}{c_2}}
    }
    <\frac{\epsilon}{8\ln(1/\epsilon)}
\end{equation}
which is less than 1 for $\epsilon<0.89$, so it is also similar to
our $(n''/2,1-\delta)$ vertex expander for the permutation model
with $0<\beta<1$, $n''=n/2$, and arbitrary small constant $\delta>0$.

\section{Connected Components and the Coin Toss Model}

\autoref{tab:models-comparison} highlights similarities between the sizes of our
expanding subgraphs in the coin toss model from~\autoref{sec:powerlaw-coin-toss-model}
and the largest connected components of power-law graphs from Aiello et al.~\cite{acl01},
although there are still some gaps.

For example, we know that the giant component exists for $\beta\in(0;3.48)$~\cite{acl01,cl06}.
%Theorem 2~\cite{acl01}, Theorem 11.2~\cite{cl06}
For $3.48<\beta<4$, the size of each connected component
is at most $\Theta\left(n^{2/\beta}\log n\right)$,
%Theorem 1~\cite{acl01}, Theorem 11.1~\cite{cl06}
and for $\beta>4$ it is $\Theta\left(n^{2/(\beta+2)}\log n\right)$.

Our proof provides a linear size edge expander only for $\beta$ up to $1.6$,
and then we have a single jump to $\Theta\left(n^{1/\beta}\right)$, which is
close to a square root of the size of the corresponding largest component.
The conclusion here is that the largest connected components
of power-law graphs have strong expansion properties,
even though their average degree $d=\omega(1)$, so they are not sparse.

\section{Coin Toss and Permutation Models}

If we consider the graphs from permutation model when $p=1$,
we notice that we get the same $(n'/2,d/2-\delta)$ edge expander
as in the coin toss model for any $0<\beta\leq 1.6$.
It demonstrates that power-law graphs with small $\beta$ possess
a similar structure regardless of whether we define
degrees or frequencies of degrees to follow a power law.
