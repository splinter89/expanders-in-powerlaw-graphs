% ~200 words, max 350 words

Random power-law graphs on $n$ vertices can be defined in different ways.
One model we study describes graphs where the expected number of vertices
of degree $x$ is proportional to a power law $1/x^\beta$, for constant $\beta>0$.
In another model, the exact degree sequence follows the power-law distribution
and each vertex $i$ has degree $pn/i^\beta$, for $0<p\leq 1$ and $\beta\geq 0$.

We show that for these models, power-law graphs contain
``large'' edge and vertex expanders.
Those are graphs in which all subsets of vertices up to a certain size have,
respectively, many outgoing edges or large vertex boundary.
%Subgraphs consist of the vertices of sufficiently large degree.
We also explore the trade-offs between expansion of the subsets
and their maximum size.

Our findings agree with and complement known results
about the presence of linear size expanders in Erd\H{o}s-R\'enyi graphs
and about the connected components of power-law graphs.

%Finally, we extend the classical result about logarithmic diameter
%of vertex expanders to the case when only small sets are expanding.
